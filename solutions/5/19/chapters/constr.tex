\begin{figure}[!ht]
\centering
\resizebox{\columnwidth}{!}{\begin{tikzpicture}
[scale=1.5,>=stealth,point/.style={draw,circle,fill = black,inner sep=0.5pt},]


%Quadrilateral sides BC, CD, AD, AB, BD
%\tikzmath{\a = 4.5; \b = 5.5; \c = 4; \d = 6; \e = 7; }
%Rotation angles DBC and ABD
%\tikzmath{\t1=51.75338012165502; \t2=34.7719440319486; }
%
\def\R{0.6227876525205392}
%Labeling points
\node (B) at (0, 0)[point,label=below left:$B$] {};
\node (C) at (9, 0)[point,label=below right:$C$] {};
\node (A) at (3,4)[point,label=above left:$A$] {};
\node (D) at (7,6)[point,label=above right:$D$] {};
\node (E) at (5.31376939,2.6568847)[point,label=above left:$E$] {};
\node (F) at (4.47213595,2.23606798)[point,label=above right:$F$] {};
\node (G) at (5.11958768,1.46028308)[point,label=above right:$G$] {};
\node (H) at (5.54592627,2.48955549)[point,label=above right:$H$] {};
\node (O) at (5.07543888,2.08150393)[point,label=right:$O$] {};
%Foot of perpendicular

\draw (A) --  node[left] {$\textrm{d}$}(B) --  node[below] {$\textrm{a}$}(C) --  node[right] {$\textrm{b}$}(D) --  node[above] {$\textrm{c}$}(A);
\draw (A) --  node[left] {$\textrm{AG}$}(G) --  node[below ,xshift=3mm] {$\textrm{DG}$}(D);
\draw (B) --  node[left] {$\textrm{BE}$}(E) --  node[below] {$\textrm{CE}$}(C);

\draw (O) circle (\R);
%Drawing and marking angles
\tkzMarkAngle[fill=orange!50,mark=||](C,B,E)
\tkzMarkAngle[fill=green!50,mark=||](E,B,A)
\tkzLabelAngle[pos=0.65](C,B,E){$\alpha$}
\tkzLabelAngle[pos=0.65](E,B,A){$\alpha$}

\end{tikzpicture}}
\caption{}
\label{fig:8.5.19_8.5.19_quadrilateral}	
\end{figure}
\item {\em Construction: } See Fig. \ref{fig:8.5.19_8.5.19_quadrilateral}.  For drawing $ABCD$, 
\begin{align}
\vec{B} &= \myvec{0\\0}
\\
\vec{A} &= d\myvec{\cos \theta\\ \sin \theta}
\\
\vec{C} &= \myvec{a \\0}
\\
\vec{D} &= \myvec{d_1 \\d_2}, \quad d_1  < a, d_2 > d \sin \theta
\end{align}
%
where 
\begin{align}
\theta = \phase{ABC}
\end{align}
%
The direction vector of the angle bisector $BE$ is given by 
\begin{align}
\frac{\vec{A}-\vec{B}}{\norm{A-B}}+\frac{\vec{C}-\vec{B}}{\norm{C-B}}
\end{align}
%
Thus, 
\begin{align}
BE:\quad \vec{x} = \vec{B}+ \lambda_1\brak{\frac{\vec{A}-\vec{B}}{\norm{A-B}}+\frac{\vec{C}-\vec{B}}{\norm{C-B}}}
\label{eq:8.5.19_constr_a}
\end{align}
Similarly,
\begin{align}
\label{eq:8.5.19_constr_b}
CE:\quad \vec{x} &= \vec{C}+ \lambda_2\brak{\frac{\vec{C}-\vec{D}}{\norm{C-D}}+\frac{\vec{C}-\vec{B}}{\norm{C-B}}}
\\
\label{eq:8.5.19_constr_c}
DG:\quad \vec{x} &= \vec{D} + \lambda_3\brak{\frac{\vec{A}-\vec{D}}{\norm{A-D}}+\frac{\vec{C}-\vec{D}}{\norm{C-D}}}
\\
\label{eq:8.5.19_constr_d}
AG:\quad \vec{x} &= \vec{A} + \lambda_4\brak{\frac{\vec{A}-\vec{D}}{\norm{A-D}}+\frac{\vec{A}-\vec{B}}{\norm{A-B}}}
\end{align}
\eqref{eq:8.5.19_constr_a}-\eqref{eq:8.5.19_constr_d} can be used to find the points $\vec{E}, \vec{F}, \vec{G}, \vec{H}$, any three of  which can be used to draw the circle and verify the fourth point lies on this circle.



