%\subsection{Area of a Circle}
%
%
\renewcommand{\theequation}{\theenumi}
\begin{enumerate}[label=\arabic*.,ref=\thesubsection.\theenumi]
\numberwithin{equation}{enumi}
%
%
\item
	Using Fig. \ref{trig_id_sin_theta}, show that 
	%
\begin{equation}
\label{trig_id_sin_theta_eq}
\sin  \theta_1 = \sin \brak{\theta_1 + \theta_2}\cos \theta_2 - \cos\brak{\theta_1+\theta_2}\sin\theta_2
\end{equation}	
	%

\begin{figure}[!ht]
	\begin{center}
		
		%\includegraphics[width=\columnwidth]{./figs/trig_id_sin_theta}
		%\vspace*{-10cm}
		\resizebox{\columnwidth}{!}{\begin{tikzpicture}
[scale =3,>=stealth,point/.style = {draw, circle, fill = black, inner sep = 1pt},]

\node (A) at (0,3)[point,label=above :$A$] {};
\node (B) at (3,0)[point,label=below :$B$] {};
\node (C) at (0,0)[point,label=below :$C$] {};
\node (D) at (0,1.5)[point,label=left :$D$] {};
\draw (A)--(B);
\draw (C)--(B);
\draw (A)--(C);
\draw (B)--(D);
\tkzMarkAngle[size=.4](A,B,D);
\tkzMarkAngle[size=.3](D,B,C);
\tkzMarkRightAngle[size=.15](A,C,B);

\node [above] at (1.6,1.5){$c$};
\node [below] at (1.6,0){$a$};
\node [below] at (1.6,1){$l$};
\node [above] at (-0.2,1.5){$b$};
\node [above] at (2.5,0){$\theta_2$};
\node [above] at (2.5,0.3){$\theta_1$};
\end{tikzpicture}}
	\end{center}
	\caption{$\sin \brak{\theta_1+\theta_2} = \sin\theta_1\cos\theta_2 + \cos\theta_1\sin\theta_2$}
	\label{trig_id_sin_theta}	
\end{figure}
%

\solution The following equations can be obtained from the figure using the forumula for the area of a triangle
%
\begin{align}
ar \brak{\Delta ABC} &= \frac{1}{2}ac \sin\brak{\theta_1 + \theta_2} \\
&= ar \brak{\Delta BDC} + ar \brak{\Delta ADB} \\
&= \frac{1}{2}cl \sin{\theta_1} + \frac{1}{2}al \sin{\theta_2} \\ 
&= \frac{1}{2}ac \sin{\theta_1} \sec \theta_2 + \frac{1}{2}a^2 \tan{\theta_2} 
\end{align}
$\brak{\because
	l = a \sec \theta_2}$.  From the above,
\begin{align}
\implies \sin\brak{\theta_1 + \theta_2} &=  \sin{\theta_1} \sec \theta_2 + \frac{a}{c} \tan{\theta_2} \\
\end{align}
\begin{multline}
\implies \sin\brak{\theta_1 + \theta_2} =  \sin{\theta_1} \sec \theta_2 
\\
+ \cos\brak{\theta_1 + \theta_2} \tan{\theta_2} 
\end{multline}
Multiplying both sides by $\cos \theta_2$,
\begin{multline}
\implies \sin\brak{\theta_1 + \theta_2}\cos{\theta_2} =  \sin{\theta_1}  
\\
+ \cos\brak{\theta_1 + \theta_2} \sin\theta_2  
\end{multline}
%
resulting in
\begin{multline}
\implies \sin \theta_1 = \sin\brak{\theta_1 + \theta_2}\cos{\theta_2} 
\\
- \cos\brak{\theta_1 + \theta_2} \sin\theta_2 
\end{multline}
\item
	Prove the following identities 
	%
	\begin{enumerate}
\item 
\begin{equation}
		\label{trig_id_sin_diff}
\sin\brak{\alpha - \beta} = \sin \alpha \cos \beta - \cos \alpha \sin \beta.
\end{equation}
\item 
\begin{equation}
\cos\brak{\alpha + \beta} = \cos \alpha \cos \beta - \sin \alpha \sin \beta.
		\label{trig_id_cos_diff}
\end{equation}

	\end{enumerate}
	%

\solution In \eqref{trig_id_sin_theta_eq}, let
%
\begin{equation}
\begin{split}
\theta_1 + \theta_2 &= \alpha \\
\theta_2 &=  \beta
\end{split}
\end{equation}
%
This gives \eqref{trig_id_sin_diff}.  In \eqref{trig_id_sin_diff}, replace $\alpha$ by 
%
$90{\degree} - \alpha$.  This results in
%
\begin{multline}
\sin\brak{90{\degree} - \alpha - \beta}
\\
=
\sin \brak{90{\degree} -\alpha} \cos \beta - \cos \brak{90{\degree} -\alpha} \sin \beta \\
\implies \cos\brak{\alpha + \beta} = \cos \alpha \cos \beta - \sin \alpha \sin \beta
\end{multline}
% 
\item
	Using \eqref{trig_id_sin_theta_eq} and \eqref{trig_id_cos_diff}, show that
\begin{align}
\label{trig_id_sin_sum}
\sin\brak{\theta_1 + \theta_2} &= \sin\theta_1  \cos\theta_2 + \cos\theta_1\sin\theta_2
\\
\cos\brak{\theta_1 - \theta_2} &= \cos\theta_1  \cos\theta_2  \sin\theta_1\sin\theta_2
\label{trig_id_cos_sum}
\end{align}

%
\solution From \eqref{trig_id_sin_theta_eq},
%
\begin{align}
 \sin \brak{\theta_1 + \theta_2}\cos \theta_2 =\sin  \theta_1 +\cos\brak{\theta_1+\theta_2}\sin\theta_2 
\end{align}
%
Using \eqref{trig_id_cos_diff} in the above,
%
\begin{multline}
\sin \brak{\theta_1 + \theta_2}\cos \theta_2 
=\sin  \theta_1 +\lbrak{\cos \theta_1\cos\theta_2 }
\\	
\rbrak{	- \sin \theta_1\sin\theta_2}\sin\theta_2 
\end{multline}
%
which can be expressed as
%
\begin{multline}
\sin \brak{\theta_1 + \theta_2}\cos \theta_2 
=\sin  \theta_1 
\\
+\cos \theta_1\cos\theta_2 \sin\theta_2 
		- \sin \theta_1\sin^2\theta_2
\end{multline}
%
Since
%
\begin{equation}
\sin^2\theta_2 = 1- \cos^2\theta_2, 
\end{equation}
%
we obtain
%
\begin{multline}
\sin \brak{\theta_1 + \theta_2}\cos \theta_2 
=\cos \theta_1\cos\theta_2 \sin\theta_2 
\\	
+ \sin \theta_1\cos^2\theta_2
\end{multline}
%
resulting in
%
\begin{equation}
\sin \brak{\theta_1 + \theta_2}
=\cos \theta_1 \sin\theta_2 
+ \sin \theta_1\cos\theta_2
\end{equation}
%
after factoring out $\cos \theta_2$.  Using a similar approach, \eqref{trig_id_cos_sum} can also be proved.
%
\item Show that 
\begin{align}
\label{eq:trig_id_sum_diff1}
\sin \theta_1 + \sin \theta_2 &= 2\sin\brak{\frac{\theta_1+\theta_2}{2}}\cos\brak{\frac{\theta_1-\theta_2}{2}}
\\
\label{eq:trig_id_sum_diff2}
\cos \theta_1 + \cos \theta_2 &= 2\cos\brak{\frac{\theta_1+\theta_2}{2}}\cos\brak{\frac{\theta_1-\theta_2}{2}}
\\
\label{eq:trig_id_sum_diff3}
\sin \theta_1 - \sin \theta_2 &= 2\sin\brak{\frac{\theta_1-\theta_2}{2}}\cos\brak{\frac{\theta_1+\theta_2}{2}}
\\
\label{eq:trig_id_sum_diff4}
\cos \theta_1 - \cos \theta_2 &= 2\sin\brak{\frac{\theta_1+\theta_2}{2}}\cos\brak{\frac{\theta_2-\theta_1}{2}}
\end{align}
%
\\
\solution Let 
%
\begin{align}
\label{eq:trig_id_ang_sum_diff}
\begin{split}
\theta_1 = \alpha + \beta
\\
\theta_2 = \alpha - \beta
\end{split}
\end{align}
%
From \eqref{trig_id_sin_sum},
%
\begin{align}
\sin \theta_1 + \sin \theta_2  &= \sin \brak{\alpha + \beta} + \sin \brak{\alpha - \beta}
\\
&= \sin \alpha \cos \beta + \cos \alpha \sin \beta 
\\
&+\sin \alpha \cos \beta - \cos \alpha \sin \beta
\\
&= 2 \sin \alpha \cos \beta
\end{align}
%
resulting in \eqref{eq:trig_id_sum_diff1}
%
\begin{align}
\because \alpha &= \frac{\theta_1 +\theta_2}{2}
\\
\beta &= \frac{\theta_1 -\theta_2}{2}
\end{align}
from \eqref{eq:trig_id_ang_sum_diff}.  Other identities may be proved similarly.
%
%\item Show that 
%\begin{align}
%\sin \brak{-\theta} = -\sin \theta
%\end{align}
%

\end{enumerate}
