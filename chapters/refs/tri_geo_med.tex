%
\subsection{Median}
\renewcommand{\theequation}{\theenumi}
\begin{enumerate}[label=\arabic*.,ref=\thesubsection.\theenumi]
\numberwithin{equation}{enumi}

\item	The line AD in Fig. \ref{fig:tri_median_def} that divides the side $a$ in two equal halfs is known as the median.

\begin{figure}[!ht]
	\begin{center}
		
		%\includegraphics[width=\columnwidth]{./figs/fig:tri_median_def}
		%\vspace*{-10cm}
%		\resizebox{\columnwidth}{!}{\begin{tikzpicture}
[scale=2,>=stealth,point/.style={draw,circle,fill = black,inner sep=0.5pt},]

\node (D) at (0, 0)[point,label=below :$D$] {};
\node (A) at (0, 3)[point,label=above :$A$]{};
\node (B) at (-3, 0)[point,label=below left:$B$]{};
\node (C) at (3, 0)[point,label=below right:$C$]{};

\draw (D)--(B);
\draw (B)--(A);
\draw (A)--(C);
\draw (C)--(D);
\draw (D)--(A);

\node [below] at (0,-0.3) {$a$};
\node [below] at (-1.5,0) {$a/2$};
\node [below] at (1.5, 0) {$a/2$};
\node [above] at (-1.5,1.5){$c$};
\node [above] at (1.5,1.5){$b$};

\end{tikzpicture}}
		\resizebox{\columnwidth}{!}{%Code by GVV Sharma
%December 8, 2019
%released under GNU GPL
%Drawing the median

\begin{tikzpicture}
[scale=2,>=stealth,point/.style={draw,circle,fill = black,inner sep=0.5pt},]

%Triangle sides
\def\a{5}
\def\b{6}
\def\c{4}
 
%Coordinates of A
\def\p{2.25}
\def\q{{sqrt(\c^2-\p^2)}}

%Labeling points
\node (A) at (\p,\q)[point,label=above right:$A$] {};
\node (B) at (0, 0)[point,label=below left:$B$] {};
\node (C) at (\a, 0)[point,label=below right:$C$] {};

%Foot of median

\node (D) at ($(B)!0.5!(C)$)[point,label=below:$D$] {};

%Drawing triangle ABC
\draw (A) -- node[left] {$\textrm{c}$} (B) -- node[below, yshift=-5mm] {$\textrm{a}$} (C) -- node[above,xshift=2mm] {$\textrm{b}$} (A);

%Drawing median AD
\draw (A) -- (D);

%
\node [below] at ($(B)!0.5!(D)$) {$\frac{a}{2}$};
\node [below] at ($(C)!0.5!(D)$) {$\frac{a}{2}$};

\end{tikzpicture}
}

	\end{center}
	\caption{Median of a Triangle}
	\label{fig:tri_median_def}	
\end{figure}
%
\item Draw Fig. \ref{fig:tri_median_def} with $a=6$, $b=5$  and $c=4$.  
\label{const:tri_median_def}
%
\\
\solution Using \eqref{eq:tri_section_formula}, since $\vec{D}$ divides $BC$ in the ratio $1:1$.
\begin{align}
\vec{D} &= \frac{\vec{B} + \vec{C}}{2}  
\end{align}
%
%The python code for  Fig. \ref{fig:tri_median_def} is
%\begin{lstlisting}
%codes/triangle/tri_median_def.py
%\end{lstlisting}
%
%and 
The  latex-tikz code is
%
\begin{lstlisting}
figs/triangle/tri_median_def.tex
\end{lstlisting}
\item
	Show that the median $AD$ in Fig. \ref{fig:tri_median_def} divides  $\Delta ABC$ into triangles $ADB$ and $ADC$ that have equal area.

\solution We have
%
\begin{align}
ar \brak{\Delta ADB} &= \frac{1}{2}\frac{a}{2}c \sin B =  \frac{1}{4}ac \sin B \\
ar \brak{\Delta ADC} &= \frac{1}{2}\frac{a}{2}b \sin C =  \frac{1}{4}ab \sin C 
\end{align}
%
Using the sine formula, $b \sin C = c \sin B$,
\begin{equation}
ar \brak{\Delta ADB} = ar \brak{\Delta ADC}
\end{equation}
\item
	$BE$ and $CF$ are the medians in Fig. \ref{ch2_median_ratio}.  Show that
	\begin{equation}
	ar \brak{\Delta BFC} = ar \brak{\Delta BEC}
	\end{equation} 
	\label{ch2_median_eq_tri}

\solution Since $BE$ and $CF$ are the medians, 

\begin{align}
ar \brak{\Delta BFC} &= \frac{1}{2} ar \brak{\Delta ABC} \\
ar \brak{\Delta BEC} &= \frac{1}{2} ar \brak{\Delta ABC} 
\end{align}
From the above, we infer that
%
\begin{equation}
ar \brak{\Delta BFC} = ar \brak{\Delta BEC}
\end{equation}


\begin{figure}[!ht]
	\begin{center}
		
		%\includegraphics[width=\columnwidth]{./figs/ch2_median_ratio}
		%\vspace*{-10cm}
		\resizebox{\columnwidth}{!}{\begin{tikzpicture}
[scale=2,>=stealth,point/.style={draw,circle,fill = black,inner sep=0.5pt},]

\node (E) at (1.5, 1.5)[point,label=above :$E$] {};
\node (F) at (-1.5, 1.5)[point,label=above :$F$] {};
\node (A) at (0, 3)[point,label=above :$A$]{};
\node (B) at (-3, 0)[point,label=below left:$B$]{};
\node (C) at (3, 0)[point,label=below right:$C$]{};
\node (a) at (0,0)[point,label=below :$a$] {};
\node (O) at (0,1)[point,label=above :$O$] {};

\draw (a)--(B);
\draw (B)--(A);
\draw (A)--(C);
\draw (C)--(a);
\draw (B)--(E);
\draw (C)--(F);
\node [above] at (-1.7,1.7) {$c$};
\node [above] at (1.7,1.7) {$b$};
\node [above] at (-2.5,.75) {$c/2$};
\node [above] at (-1,2.1) {$c/2$};
\node [above] at (2.5,.75) {$b/2$};
\node [above] at (1,2.1) {$b/2$};
%\node [above] at (1,1.3) {$p$};
%\node [above] at (-1,1.3) {$q$};

\tkzMarkAngle[size=.3](F,O,B);
\tkzMarkAngle[size=.3](C,O,E);
\draw (-0.5,1) node{$\theta$};
\draw (0.5,1) node{$\theta$};

\end{tikzpicture}}
	\end{center}
	\caption{$O$ is the Intersection of Two Medians}
	\label{ch2_median_ratio}	
\end{figure}
%
%\item
%	The medians $BE$ and $CF$ in Fig. \ref{ch2_median_ratio} meet at point $O$.  Show that
%	\begin{equation}
%	\frac{OB}{OE} = \frac{OC}{OF} 
%	\end{equation} 
%
%\solution From Problem \ref{ch2_median_eq_tri},
%
%\begin{equation}
%ar \brak{\Delta BFC} = ar \brak{\Delta BEC}
%\end{equation}
%%
%\begin{multline}
%\Rightarrow ar \brak{\Delta BOF} + ar \brak{\Delta BOC} \\
%= ar \brak{\Delta BOC} + ar \brak{\Delta COE} 
%\end{multline}
%resulting in
%%
%\begin{equation}
%ar \brak{\Delta BOF} 
%=  ar \brak{\Delta COE} 
%\end{equation}
%%
%Using the sine formula for area of a triangle, the above equation can be expressed as
%%
%\begin{align}
%\frac{1}{2}OB\,OF \sin \theta &= \frac{1}{2}OC\,OE \sin \theta \\
%\Rightarrow 	\frac{OB}{OE} = \frac{OC}{OF} 
%\end{align}
%
\item
	We know that the median of a triangle  divides it into two triangles with equal area. Using this result along with the sine formula for the area of a triangle in Fig. \ref{ch2_supp_sin},
\begin{figure}[!ht]
	\begin{center}
		
		%\includegraphics[width=\columnwidth]{./figs/ch2_supp_sin}
		%\vspace*{-10cm}
		\resizebox{\columnwidth}{!}{\begin{tikzpicture}
[scale=2,>=stealth,point/.style={draw,circle,fill = black,inner sep=0.5pt},]



\node (A) at (0, 3)[point,label=above :$A$]{};
\node (B) at (-3, 0)[point,label=below left:$B$]{};
\node (C) at (3, 0)[point,label=below right:$C$]{};
\node (D) at (0,0)[point,label=below :$D$] {};
%\node (a) at (0,-0.2) [point,label=below :$a$] {};


\draw (B)--(A);
\draw (A)--(C);
\draw (C)--(B);
\draw (A)--(D);

\node [above] at (-1.5,-.7) {$a/2$};
\node [above] at (1.5,-.7) {$a/2$};
\node [above] at (-1.7,1.7) {$c$};
\node [above] at (1.7,1.7) {$b$}; 
\node [below] at (0,-0.3) {$a$}; 

\tkzMarkAngle[size=.3](A,D,B);
\tkzMarkAngle[size=.4](C,D,A);
\draw (-.5,0.1) node{$\theta$};
\draw (.7,0.1) node{180 - $\theta$};

\end{tikzpicture}}
	\end{center}
	\caption{$\sin \theta = \sin \brak{180^{\degree}-\theta}$}
	\label{ch2_supp_sin}	
\end{figure}

\begin{align}
\frac{1}{2}\frac{a}{2}AD \sin \theta &= \frac{1}{2}\frac{a}{2}AD \sin \brak{180^{\degree}-\theta} \\
\Rightarrow \sin \theta &= \sin \brak{180^{\degree} - \theta}.
\label{sin_supp}
\end{align}
Note that our geometric definition of $\sin \theta$ holds only for $\theta < 90^{\degree}$.  \eqref{sin_supp} allows us to extend this definition for $\angle ADC > 90^{\degree}$.

%\end{enumerate}

%\subsection{Similar Triangles}
%\renewcommand{\theequation}{\theenumi}
%\begin{enumerate}[label=\arabic*.,ref=\thesubsection.\theenumi]
%\numberwithin{equation}{enumi}

%
%
\item
	In Fig. \ref{ch2_sim_triang}, show that $EF = \frac{a}{2}$.  

%
\begin{figure}[!ht]
	\begin{center}
		
		%\includegraphics[width=\columnwidth]{./figs/ch2_median_ratio_val}
		%\vspace*{-10cm}
		\resizebox{\columnwidth}{!}{\begin{tikzpicture}
[scale=2,>=stealth,point/.style={draw,circle,fill = black,inner sep=0.5pt},]

\node (E) at (1.5, 1.5)[point,label=above :$E$] {};
\node (F) at (-1.5, 1.5)[point,label=above :$F$] {};
\node (A) at (0, 3)[point,label=above :$A$]{};
\node (B) at (-3, 0)[point,label=below left:$B$]{};
\node (C) at (3, 0)[point,label=below right:$C$]{};
\node (a) at (0,0)[point,label=below :$a$] {};
\node (O) at (0,1)[point,label=below :$O$] {};

\draw (a)--(B);
\draw (B)--(A);
\draw (A)--(C);
\draw (C)--(a);
\draw (B)--(E);
\draw (C)--(F);
\draw (E)--(F);
\node [above] at (-1.7,1.7) {$c$};
\node [above] at (1.7,1.7) {$b$};
\node [above] at (-2.5,.75) {$c/2$};
\node [above] at (-1,2.1) {$c/2$};
\node [above] at (2.5,.75) {$b/2$};
\node [above] at (1,2.1) {$b/2$};
\node [below] at (1,1.3) {$p$};
\node [below] at (-1,1.3) {$q$};
\node [above] at (-.7,.5) {$k_{p}$};
\node [above] at (.7,.5) {$k_{q}$};

\tkzMarkAngle[size=.3](F,O,B);
\tkzMarkAngle[size=.3](C,O,E);
\draw (-0.5,1) node{$\theta$};
\draw (0.5,1) node{$\theta$};

\end{tikzpicture}}
	\end{center}
	\caption{Similar Triangles}
	\label{ch2_sim_triang}	
\end{figure}

\solution Using the cosine formula for $\Delta AEF$,
%
\begin{align}
EF^2 &= \brak{\frac{b}{2}}^2 + \brak{\frac{c}{2}}^2 - 2 \brak{\frac{b}{2}}\brak{\frac{c}{2}} \cos A \\
&= \frac{b^2 + c^2 - 2bc \cos A}{4} \\
&= \frac{a^2}{4} \\
\Rightarrow EF &= \frac{a}{2}
\end{align}
%

\item
	The ratio of sides of triangles AEF and ABC is the same.  Such triangles are known as similar triangles.

\item
	Show that similar triangles have the same angles.

\solution Use cosine formula and the proof is trivial.
\item	Show that in Fig. \ref{ch2_sim_triang}, $EF || BC$.

\solution Since $\Delta AEF \sim \Delta ABC$, $\angle AEF = \angle ACB$.  Hence the line $EF||BC$
%
\item	Show that $\triangle OEF \sim \triangle OEC$.

\item Show that

%	Using Fig. \ref{ch2_median_ratio_val}, 
	\begin{align}
	\frac{OB}{OE} = \frac{OC}{OF} = 2
	\end{align}

%\item
%	In Fig. \ref{ch2_median_ratio_val}, show that
%	\begin{equation}
%	ar\brak{\Delta AFE} = 	\frac{1}{4} ar\brak{\Delta ABC}
%	\end{equation}
%	\label{ch2_small_triang}
%
%\solution We have
%%
%\begin{align}
%ar\brak{\Delta AFE} &= \frac{1}{2} \frac{b}{2}\frac{c}{2}\sin A = \frac{1}{4}.\frac{1}{2}bc \sin A \\
%&=\frac{1}{4}ar\brak{\Delta ABC}
%\end{align}
%
		%\includegraphics[width=\columnwidth]{./figs/ch2_median_ratio_val}
		%\vspace*{-10cm}

%
%\begin{figure}[!ht]
%	\begin{center}
%		
%		\resizebox{\columnwidth}{!}{\begin{tikzpicture}
[scale=2,>=stealth,point/.style={draw,circle,fill = black,inner sep=0.5pt},]

\node (E) at (1.5, 1.5)[point,label=above :$E$] {};
\node (F) at (-1.5, 1.5)[point,label=above :$F$] {};
\node (A) at (0, 3)[point,label=above :$A$]{};
\node (B) at (-3, 0)[point,label=below left:$B$]{};
\node (C) at (3, 0)[point,label=below right:$C$]{};
\node (a) at (0,0)[point,label=below :$a$] {};
\node (O) at (0,1)[point,label=below :$O$] {};

\draw (a)--(B);
\draw (B)--(A);
\draw (A)--(C);
\draw (C)--(a);
\draw (B)--(E);
\draw (C)--(F);
\draw (E)--(F);
\node [above] at (-1.7,1.7) {$c$};
\node [above] at (1.7,1.7) {$b$};
\node [above] at (-2.5,.75) {$c/2$};
\node [above] at (-1,2.1) {$c/2$};
\node [above] at (2.5,.75) {$b/2$};
\node [above] at (1,2.1) {$b/2$};
%\node [below] at (1,1.3) {$p$};
%\node [below] at (-1,1.3) {$q$};
%\node [above] at (-.7,.5) {$k_{p}$};
%\node [above] at (.7,.5) {$k_{q}$};

\tkzMarkAngle[size=.3](F,O,B);
\tkzMarkAngle[size=.3](C,O,E);
\draw (-0.5,1) node{$\theta$};
\draw (0.5,1) node{$\theta$};

\end{tikzpicture}}
%	\end{center}
%	\caption{$O$ divides medians in the ratio $2:1$}
%	\label{ch2_median_ratio_val}	
%\end{figure}
%

%\solution Using the sine formula and \eqref{ch2_small_triang}, areas of some triangles in Fig. \ref{ch2_median_ratio_val} are listed in the following table
%\begin{center}
%\begin{tabular}{|c|c|}
%	\hline  Triangle& Area  \\ 
%	\hline  OFE & $\frac{1}{2}pq \sin \theta $ \\ 
%	\hline  BOF &  $\frac{k}{2}pq \sin \theta$\\ 
%	\hline  COE & $\frac{k}{2}pq \sin \theta$ \\ 
%	\hline  BOC&  $\frac{k^2}{2}pq \sin \theta$\\
%	\hline  BOC&  $\frac{1}{2} ar \brak{\Delta ABC}$\\  
%	\hline 
%\end{tabular} 
%\end{center}
%
%where we have used the fact that 
%
%\begin{equation}
%\sin \angle BOC = \sin \brak{180^{\degree}-\theta} = \sin \theta
%\end{equation}
%%
%Since $BE$ is the median
%\begin{align}
%\begin{split}
%ar\brak{\Delta BEC} &= \frac{1}{2} ar\brak{\Delta ABC}  
%\\
%& = ar\brak{\Delta BOC} + ar\brak{\Delta COE}\\
%ar\brak{\Delta BEA} &= \frac{1}{2} ar\brak{\Delta ABC}
%\\
%& = ar\brak{\Delta AFE} + ar\brak{\Delta BOF} +  ar\brak{\Delta FOE}  
%\end{split}
%\end{align}
%
%Substituting the respective values from the above table,
%\begin{align}
%\begin{split}
%\frac{1}{2} ar\brak{\Delta ABC} &= \frac{k^2}{2}pq \sin \theta + \frac{k}{2}pq \sin \theta
%\\
%\frac{1}{2} ar\brak{\Delta ABC} &= \frac{k}{2}pq \sin \theta + \frac{1}{2}pq \sin \theta + \frac{1}{4} ar\brak{\Delta ABC}
%\end{split}
%\end{align}
%%
%Simplifying the above,
%%
%\begin{align}
%k(k+1) = 2(k+1)
%\end{align}
%%
%Since $k \neq -1$, $k = 2$ and the proof is complete.
%
%
%\item
%	In Fig. \ref{ch2_median_3}, the line $AO$ cuts $EF$ at $G$ and is extended to meet the side $BC$ at $D$.  Show that 
%	%
%	\begin{equation}
%	\frac{OA}{OD} = 2.
%	\end{equation}
%	%  
%
%\begin{figure}[!ht]
%	\begin{center}
%		
%		%\includegraphics[width=\columnwidth]{./figs/ch2_median_3}
%		%\vspace*{-10cm}
%		\resizebox{\columnwidth}{!}{\begin{tikzpicture}
[scale=2,>=stealth,point/.style={draw,circle,fill = black,inner sep=0.5pt},]

\node (E) at (1.5, 1.5)[point,label=above :$E$] {};
\node (F) at (-1.5, 1.5)[point,label=above :$F$] {};
\node (A) at (0, 3)[point,label=above :$A$]{};
\node (B) at (-3, 0)[point,label=below left:$B$]{};
\node (C) at (3, 0)[point,label=below right:$C$]{};
\node (O) at (0,1)[point,label=below right :$O$] {};
\node (D) at (0,0)[point,label=below :$D$] {};

\draw (B)--(A);
\draw (A)--(C);
\draw (C)--(B);
\draw (B)--(E);
\draw (C)--(F);
\draw (E)--(F);
\draw (A)--(D);

\node [above] at (-1.7,1.7) {$c$};
\node [above] at (1.7,1.7) {$b$};
\node [above] at (-2.5,.75) {$c/2$};
\node [above] at (-1,2.1) {$c/2$};
\node [above] at (2.5,.75) {$b/2$};
\node [above] at (1,2.1) {$b/2$};
\node [below] at (1,1.3) {$p$};
\node [below] at (-1,1.3) {$q$};
\node [above] at (-.7,.5) {$2p$};
\node [above] at (.7,.5) {$2q$};
\node [above] at (0,-.5) {$a$};
\node [above] at (0,1.9) [point,label=above right :$r$] {};
\node (G) at (0,1.5)[point,label=above right :$G$] {};

\tkzMarkAngle[size=.3](F,O,B);
\tkzMarkAngle[size=.3](C,O,E);
\draw (-0.5,1) node{$\theta$};
\draw (0.5,1) node{$\theta$};

\end{tikzpicture}}
%	\end{center}
%	\caption{Similar Triangles and Median}
%	\label{ch2_median_3}	
%\end{figure}
%
%\solution Since $EF || BC$,
%%
%\begin{equation}
%\Delta AGE \sim \Delta ADC \Rightarrow AG = GD = \frac{AD}{2}
%\end{equation}
%%
%Also,
%\begin{align}
%\Delta OGE \sim \Delta ODB \Rightarrow OD = 2r
%\end{align}
%Thus, we have the following relations
%\begin{align}
%GD &= OG + OD  = r + 2r = 3r\\
%AG &= GD = 3r\\
%OA &= OG + AG = r + 3r = 4r
%\end{align}
%Hence
%%
%\begin{equation}
%\frac{OA}{OD} = \frac{4r}{2r} = 2
%\end{equation}
%%
%
%\item
%	In Fig. \ref{ch2_median_final},$BE$ is a median of $\Delta ABC$ and $\frac{OB}{OE} = 2$.  Show that AD is also a median. 
%
%%
%\begin{figure}[!ht]
%	\begin{center}
%		
%		%\includegraphics[width=\columnwidth]{./figs/ch2_median_final}
%		%\vspace*{-10cm}
%		\resizebox{\columnwidth}{!}{\begin{tikzpicture}
[scale=2,>=stealth,point/.style={draw,circle,fill = black,inner sep=0.5pt},]

\node (D) at (1, 0)[point,label=below :$D$] {};
\node (A) at (0, 3)[point,label=above :$A$]{};
\node (B) at (-3, 0)[point,label=below left:$B$]{};
\node (C) at (3, 0)[point,label=below right:$C$]{};
\node (E) at (1.5, 1.5)[point,label=above right:$E$]{};
\node (O) at (.6, 1.2)[point,label=above right:$O$]{};

\draw (D)--(B);
\draw (B)--(A);
\draw (A)--(C);
\draw (C)--(D);
\draw (D)--(A);
\draw (D)--(E);
\draw (E)--(B);
%\tkzMarkRightAngle[size=.2](A,D,C)

\node [below] at (0,-0.1) {$x$};
\node [below] at (2,-0.1) {$a-x$};
\node [above] at (-1.5,1.5){$c$};
\node [above] at (1.7,1.7){$b$};
\node [above] at (.6,.5){$q$};
\node [above] at (.16,2){$kq$};
\node [above] at (-.7,.4){$2p$};
\node [above] at (1.2,1.1){$p$};
\node [above] at (.9,2.2){$b/2$};
\node [above] at (2.3,0.8){$b/2$};

\tkzMarkAngle[size=.2](A,O,B);
\draw (0.3,1.3) node{$\alpha$};
\tkzMarkAngle[size=.2](D,O,E);
\draw (0.9,1.1) node{$\alpha$};
\tkzMarkAngle[size=.4](D,B,O);
\draw (-2.3,0.1) node{$\theta_1$};
\tkzMarkAngle[size=.5](O,B,A);
\draw (-2.3,0.4) node{$\theta_2$};
%\tkzMarkAngle[size=.2](E,B,A);
%\draw (-3.1,0.3) node{$\theta_2$};

\end{tikzpicture}}
%	\end{center}
%	\caption{Medians meet at a point}
%	\label{ch2_median_final}	
%\end{figure}
%
%\solution Since BE is a median, the areas of  triangles  $\triangle BEC$ and $\triangle BEA$ are equal. Hence,
%\begin{align}
%\frac{1}{2}BE.BC\sin \theta_1 &= \frac{1}{2}BE.AB\sin \theta_2
%\\
%\implies a\sin \theta_1 &= c\sin \theta_2
%\label{eq:med_conv_area}
%\end{align}
%%
%Using the sine formula in $\triangle OBD$ and $\triangle BOA$,
%\begin{align}
%\label{eq:med_conv_sin1}
%\frac{q}{\sin \theta_1} &= \frac{x}{\sin \alpha} 
%\\
%\frac{c}{\sin \alpha} &= \frac{kq}{\sin \theta_2}
%\label{eq:med_conv_sin2}
%\end{align}
%Mutiplying \eqref{eq:med_conv_area},\eqref{eq:med_conv_sin1}
%and \eqref{eq:med_conv_sin2},
%%
%we obtain
%\begin{align}
%x = \frac{a}{k}
%\label{eq:med_conv_ak}
%\end{align}
%after simplification.
%Let the area of $\triangle ABC = \Delta$. Then, in terms of area,
%\begin{align}
%\triangle ABD &= \triangle OBD + \triangle OBA
%\\
%\implies \frac{1}{2}(c)(x)\sin B &=\frac{1}{2}(2p)(q)\sin \alpha 
%\\
%&\quad + \frac{1}{2}(2p)(kq)\sin \alpha
%\\
%\implies k\brak{k+1}pq \sin \alpha  &= \frac{1}{2}ca \sin B = \Delta
%\label{eq:med_conv_del1}
%\end{align}
%%
%Similarly,
%\begin{align}
%\triangle ABE &= \triangle AOB + \triangle AOE
%\\
%\implies \frac{\Delta}{2} &=\frac{1}{2}(2p)(kq)\sin \alpha 
%\\
%&\quad + \frac{1}{2}(p)(kq)\sin \alpha
%\\
%\implies 3kpq \sin \alpha  &=  \Delta
%\label{eq:med_conv_del2}
%\end{align}
%%
%after simplification. From \eqref{eq:med_conv_del1}
%and \eqref{eq:med_conv_del2},
%%
%\begin{align}
%k\brak{k+1}pq \sin \alpha &=  3kpq \sin \alpha  
%\\
%\implies k = 2.
%\end{align}
%Thus, 
%\begin{align}
%BD &= DC
%\\
%\frac{OA}{OD} &= 2.
%\end{align}
%Thus, $AD$ is a median.
%\\
%{\em Conclusion:} 
\item Show that the medians of a triangle meet at a point.
\end{enumerate}
