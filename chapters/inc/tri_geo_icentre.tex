%%
%\subsection{Perpendicular Bisectors}
\renewcommand{\theequation}{\theenumi}
\begin{enumerate}[label=\arabic*.,ref=\thesubsection.\theenumi]
\numberwithin{equation}{enumi}

\item Find a point $\vec{O}$ that is equidistant from the sides of $\triangle ABC$ for $a = 5, b = 6, c = 4$. Here, distance means the perpendicular distanace.
%
\solution Let $\vec{I}$ be the desired point and  $\vec{D}, \vec{E}, \vec{F}$ are on  $BC, CA, AB$ such that $ID \perp BC, IE \perp CA, IF \perp AB$ and $ID = IE = IF = r$, then, applying \eqref{eq:tri_baudh} in $\triangle s$ $IDB$ and $IEB$,
\begin{align}
\label{eq:tri_icentre_baudhd}
\begin{split}
IB^2 &= ID^2+BD^2 = r^2 + BD^2 
\\
IB^2 &= IF^2+BF^2 = r^2 + BF^2
\end{split}
\end{align}
From the above, it is obvious that $BD = BF$. Similarly, $AE = AF, CD = CF$.  Denoting these lengths as $x, y$ and $z$, as shown in Fig. \ref{fig:tri_icentre},
%
\begin{align}
x + y = a
y+z = b
x + z = c
\end{align}
%
which can be expressed as the matrix equation
%
\begin{align}
\myvec{
1 & 1 & 0
\\
0 & 1 & 1
\\
1 & 0 & 1
}
\myvec{x \\ y \\ z}
=
\myvec{a\\b\\c}
\label{eq:tri_icentre_mat}
\end{align}
%
Section formula can be used to compute 
\begin{align}
\vec{D} = \frac{x\vec{C}+y\vec{B}}{x+y}
\\
\vec{E} = \frac{y\vec{A}+z\vec{C}}{y+z}
\\
\vec{F} = \frac{z\vec{B}+x\vec{A}}{z+x}
\end{align}
%
Note that $\vec{I}$ is the circumcentre of $\triangle DEF$.  Thus, \eqref{eq:circle_const_chord_mat} can be used to compute $\vec{I}$.  
%
This is done by the python code below
%
\begin{lstlisting}
codes/circle/tri_icentre.py
\end{lstlisting}
%
and the equivalent latex-tikz code to draw Fig. \ref{fig:tri_icentre} is
%
\begin{lstlisting}
figs/circle/tri_icentre.tex
\end{lstlisting}
%
\begin{figure}[!ht]
	\begin{center}
		
		\resizebox{\columnwidth}{!}{%Code by GVV Sharma
%December 10, 2019
%released under GNU GPL
%Locating the incentre

\begin{tikzpicture}
[scale=2,>=stealth,point/.style={draw,circle,fill = black,inner sep=0.5pt},]

%Triangle sides
\def\a{5}
\def\b{6}
\def\c{4}
 
%Coordinates of A
%\def\p{{\a^2+\c^2-\b^2}/{(2*\a)}}
\def\p{0.5}
\def\q{{sqrt(\c^2-\p^2)}}

%Labeling points
\node (A) at (\p,\q)[point,label=above right:$A$] {};
\node (B) at (0, 0)[point,label=below left:$B$] {};
\node (C) at (\a, 0)[point,label=below right:$C$] {};

%Circumcentre

\node (I) at (1.5,1.32287566)[point,label=right:$I$] {};
\node (D) at (1.5,0)[point,label=below:$D$] {};
\node (E) at (2.375,2.3150324)[point,label=above right:$E$] {};
\node (F) at (0.1875,1.48823511)[point,label=left:$F$] {};

%Drawing triangle ABC
\draw (A) -- node[left] {$\textrm{c}$} (B) -- node[below] {$\textrm{a}$} (C) -- node[above,yshift=2mm] {$\textrm{b}$} (A);
%Drawing OA, OB, OC
\draw (I) --  (A);
\draw (I) --  (B);
\draw (I) --  (C);

%Drawing OD, OE, OF
\draw (I) -- node[right] {$\textrm{r}$} (D);
\draw (I) -- node[below] {$\textrm{r}$} (E);
\draw (I) -- node[below] {$\textrm{r}$} (F);

\tkzMarkAngle[fill=green!60,size=.3](I,B,F)
\tkzMarkAngle[fill=green!40,size=.3](D,B,I)
%
%
\tkzMarkAngle[fill=red!60,size=.3](F,A,I)
\tkzMarkAngle[fill=red!40,size=.3](I,A,E)


\tkzMarkAngle[fill=orange!60](E,C,I)
\tkzMarkAngle[fill=orange!40](I,C,D)
%
\tkzMarkRightAngle[fill=blue!20,size=.3](A,E,I)
\tkzMarkRightAngle[fill=blue!20,size=.3](A,F,I)
\tkzMarkRightAngle[fill=blue!20,size=.3](I,D,C)

%Labeling x,y,z
\node (x1) at ($(B)!0.5!(D)$)[label=below:$x$] {};
\node (x2) at ($(B)!0.5!(F)$)[label=left:$x$] {};
\node (y1) at ($(C)!0.5!(D)$)[label=below:$y$] {};
\node (y2) at ($(C)!0.5!(E)$)[label=right:$y$] {};
\node (z1) at ($(A)!0.5!(E)$)[label=right:$z$] {};
\node (z2) at ($(A)!0.5!(F)$)[label=left:$z$] {};


\end{tikzpicture}
}
	\end{center}
	\caption{Incentre $I$ of $\triangle ABC$}
	\label{fig:tri_icentre}	
\end{figure}
%
\item $r$ is known as the {\em inradius} of $\triangle ABC$.  Find $r$ for  the given values of $a,b,c$.
\\
\solution From Fig. \ref{fig:tri_icentre}
%
\begin{align}
\because ar\brak{ABC} &= ar\brak{IBC}+ar\brak{ICA}+ar\brak{IAB}
\\
&=\frac{1}{2}ra+\frac{1}{2}rb+\frac{1}{2}rc = \frac{a+b+c}{2}r,
\\
r &= \frac{\sqrt{s\brak{s-a}\brak{s-b}\brak{s-c}}}{s}
\end{align}
%
using Hero's formula.
%
The following python code computes the {\em inradius}
%
\begin{lstlisting}
codes/circle/tri_iradius.py
\end{lstlisting}

\end{enumerate}

