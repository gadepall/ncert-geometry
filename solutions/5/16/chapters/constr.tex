
%\renewcommand{\thefigure}{\theenumi.\arabic{figure}}
\begin{figure}[!ht]
\centering
\resizebox{\columnwidth}{!}{\begin{tikzpicture}
[scale =2,>=stealth,point/.style = {draw, circle, fill = black, inner sep = 1pt},]

\def\rad{2}
\coordinate [point, label={below: $O$ }] (O) at (0, 0);
\draw (O) circle (\rad);
\node (B) at (-2,0)[point,label=left :$B$] {};
\node (A) at (2,0)[point,label= right :$A$] {};
\node (C) at (1.414, 1.414)[point,label=right :$C$] {};
\node (D) at (-0.518, 1.932)[point,label= above left :$D$] {};
\node (E) at (0.597,3.385)[point,label= above :$E$] {};
\draw (A)--(B);
\draw (C)--(A);
\draw (B)--(D);
\draw (B)--(E);
\draw (C)--(D);
\draw (E)--(A);

\draw [thick,dashed](C)--(B);
\draw [thick,dashed](O)--(D);
\draw [thick,dashed](O)--(C);
\draw [thick,dashed](A)--(D);
\tkzMarkRightAngle[size=0.2](B,C,A)
\tkzMarkRightAngle[size=0.2](B,D,A)
\tkzMarkAngle[fill=orange!40,size=0.3cm,mark=](C,O,D)
\tkzLabelAngle[pos=0.4](C,O,D){$\beta$}
\tkzMarkAngle[fill=orange!40,size=0.5cm,mark=](C,A,D)
\tkzLabelAngle[pos=0.6](C,A,D){$\gamma$}
\tkzMarkAngle[fill=orange!40,size=0.3cm,mark=](A,O,C)
\tkzLabelAngle[pos=0.4](A,O,C){$\theta$}
\tkzMarkAngle[fill=orange!40,size=0.6cm,mark=](A,O,D)
\tkzLabelAngle[pos=0.7](A,O,D){$\theta_1$}

\node [above] at (-1,-0.2){$r$};
\node [above] at (1,-0.2){$r$};
\node [above] at (0.7,0.4){$r$};
\node [above] at (-0.5 , 1){$r$};
\node [above] at (0.3 ,1.47){$r$};
%\node [above right] at (0.4,-0.5){$\theta=\frac{2\pi}{n}$};
\end{tikzpicture}}
\caption{}
\label{fig:8.5.16_circle_latex}	
\end{figure}
%
%
%\renewcommand{\thefigure}{\theenumi}
%
\item {\em Construction: }In Fig. \ref{fig:8.5.16_circle_latex} the known parameters are
%\label{}
\\
%
%\solution From the given information, 
%$\triangle ABC$ are 
\begin{align}
\vec{O} &= \myvec{0\\0} ,
\\
\vec{A} &= \myvec{r\\0} ,
\label{eq:8.5.16_constr_a}
\\
 \vec{B} &= \myvec{-r\\0} 
\label{eq:8.5.16_constr_b}
\\
\vec{C}&= r\myvec{\cos\theta\\\sin\theta}
\label{eq:8.5.16_constr_cgen}
\end{align}
%
Let 
\begin{align}
\vec{D} & = r\myvec{\cos\theta_1\\\sin\theta_1} 
\label{eq:8.5.16_constr_dgen}
\end{align}
From the given information,
\begin{align}
 \norm{\vec{D} - \vec{C}} &= r
\\
\implies \brak{\vec{D} - \vec{C}}^T\brak{\vec{D} - \vec{C}} &= r^2\\
 \label{eq:8.5.16_dist_formula}
\\
\implies  \norm{D} ^2 - 2\vec{D}^T\vec{C} +  \norm{C}^2 &=r^2 
\end{align}
In the above, 
\begin{align}
\because \norm{D} &= \norm{\vec{C}} = r,
\\
\frac{\vec{D}^T\vec{C}}{\norm{D}\norm{C}}  &= \frac{1}{2}
\\
\implies \cos \brak{\theta_1-\theta} &= 60 \degree
\label{eq:8.5.16_theta_diff}
\end{align}
using the definition of the inner product.  $\because \theta$ is known, we get $\theta_1$ from \eqref{eq:8.5.16_theta_diff}
and $\vec{D}$ from 
\eqref{eq:8.5.16_constr_dgen}. 
%
\subitem Thus,
\begin{align}
BD: \vec{x} &= \vec{B} + \lambda_1 \brak{\vec{B}-\vec{D}}
\\
AC: \vec{x} &= \vec{A} + \lambda_2 \brak{\vec{A}-\vec{C}}
\end{align}
%
which can be used to obtain $\vec{E}$.

