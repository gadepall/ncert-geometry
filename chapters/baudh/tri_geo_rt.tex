\renewcommand{\theequation}{\theenumi}
\begin{enumerate}[label=\arabic*.,ref=\thesubsection.\theenumi]
\numberwithin{equation}{enumi}
\item


A right angled triangle looks like Fig. \ref{fig:tri_right_angle}.
\begin{figure}[!ht]
\centering
\resizebox{\columnwidth}{!}{%Code by GVV Sharma
%December 6, 2019
%released under GNU GPL
%Drawing a right angled triangle

\begin{tikzpicture}[scale=2]

%Triangle sides
\def\a{4}
\def\c{3}

%Marking coordiantes
\coordinate [label=above:$A$] (A) at (0,\c);
\coordinate [label=left:$B$] (B) at (0,0);
\coordinate [label=right:$C$] (C) at (\a,0);

%Drawing triangle ABC
\draw (A) -- node[left] {$\textrm{c}$} (B) -- node[below] {$\textrm{a}$} (C) -- node[above,,xshift=2mm] {$\textrm{b}$} (A);

%Drawing and marking angles
\tkzMarkAngle[fill=orange!40,size=0.5cm,mark=](A,C,B)
\tkzMarkRightAngle[fill=blue!20,size=.3](A,B,C)
\tkzLabelAngle[pos=0.65](A,C,B){$\theta$}
\end{tikzpicture}
}
\caption{Right Angled Triangle}
\label{fig:tri_right_angle}	
\end{figure}
with angles $\angle A,\angle B$ and $\angle C$ and sides $a, b$ and $c$.  The unique feature of this triangle is $\angle B$ which is defined to be $90\degree$.
\item
	For simplicity, let the greek letter $\theta = \angle C$.  We have the following definitions.
\begin{equation}
\label{eq:tri_trig_defs}
\begin{matrix}
	\sin \theta = \frac{c}{b} & 	\cos \theta = \frac{a}{b} \\
	\tan \theta = \frac{c}{a} & \cot \theta = \frac{1}{\tan \theta} \\
	\csc \theta = \frac{1}{\sin \theta} & \sec \theta = \frac{1}{\cos \theta}
	\end{matrix}
\end{equation}
%
\item Draw Fig. \ref{fig:tri_right_angle} for $a = 4, c =3$.
\label{const:tri_right_angle}
%
\\
\solution The vertices of $\triangle ABC$ are 
\begin{align}
\vec{A} = \myvec{0\\c} = \myvec{0\\3}, \vec{B} = \myvec{0\\0}, \vec{C} = \myvec{a\\0}=\myvec{4\\0}
\end{align}
%
The python code for  Fig. \ref{fig:tri_right_angle} is
\begin{lstlisting}
codes/triangle/tri_right_angle.py
\end{lstlisting}
%
and the equivalent latex-tikz code is
%
\begin{lstlisting}
figs/triangle/tri_right_angle.tex
\end{lstlisting}
%
The above latex code can be compiled as a standalone document as
%
\begin{lstlisting}
figs/triangle/tri_right_angle_alone.tex
\end{lstlisting}
%

\item Draw Fig. \ref{fig:tri_polar} for $a = 4, c =3$.
\label{const:tri_polar}
%
\\
\solution The vertices of $\triangle ABC$ are 
\begin{align}
\vec{A} = \myvec{a\\c} = \myvec{4\\3}, \vec{B} = \myvec{a\\0}  = \myvec{4\\0}, \vec{C} = \myvec{0\\0}.
\end{align}
%
The python code for  Fig. \ref{fig:tri_polar} is
\begin{lstlisting}
codes/triangle/tri_polar.py
\end{lstlisting}
%
and the equivalent latex-tikz code is
%
\begin{lstlisting}
figs/triangle/tri_polar.tex
\end{lstlisting}
\begin{figure}[!ht]
\centering
\resizebox{\columnwidth}{!}{%Code by GVV Sharma
%December 6, 2019
%released under GNU GPL
%Drawing a right angled triangle

\begin{tikzpicture}[scale=2]

%Triangle sides
\def\a{4}
\def\c{3}

%Marking coordiantes
\coordinate [label=above:$A$] (A) at (\a,\c);
\coordinate [label=below:$B$] (B) at (\a,0);
\coordinate [label=left:$C$] (C) at (0,0);

%Drawing triangle ABC
\draw (A) -- node[left] {$\textrm{c}$} (B) -- node[below] {$\textrm{a}$} (C) -- node[above left,xshift=2mm] {$\textrm{b}$} (A);

%Drawing and marking angles
\tkzMarkAngle[fill=orange!40,size=0.5cm,mark=](B,C,A)
\tkzMarkRightAngle[fill=blue!20,size=.3](A,B,C)
\tkzLabelAngle[pos=0.65](A,C,B){$\theta$}
\end{tikzpicture}
}
\caption{Right Angled Triangle}
\label{fig:tri_polar}	
\end{figure}
%
\item The vertex  $\vec{A}$ can also be expressed  in {\em polar coordinate form} as
\label{prob:tri_polar}
%
\begin{align}
\vec{A} = \myvec{b\cos \theta\\ b \sin \theta} 
\end{align}
%

\end{enumerate}

